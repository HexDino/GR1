%\makeglossaries
\makenoidxglossaries

% Danh mục thuật ngữ và từ viết tắt
\newglossaryentry{iaas}{
    type=\acronymtype,
    name={IaaS},
    description={Dịch vụ hạ tầng},
    first={Dịch vụ hạ tầng (Infrastructure  as  a  Service - IaaS)}
}
\newglossaryentry{API}{
    type=\acronymtype,
    name={API},
    description={Giao diện lập trình ứng dụng (Application Programming Interface)},
    first={API}
}
\newglossaryentry{EUD}{
    type=\acronymtype,
    name={EUD},
    description={Phát triển ứng dụng người dùng cuối(End-User Development)},
    first={End-User Development}
}
\newglossaryentry{GWT}{
    type=\acronymtype,
    name={GWT},
    description={Công cụ lập trình Javascript bằng Java của Google (Google Web Toolkit)},
    first={Công cụ lập trình Javascript bằng Java của Google (Google Web Toolkit}
}
\newglossaryentry{HTML}{
    type=\acronymtype,
    name={HTML},
    description={Ngôn ngữ đánh dấu siêu văn bản (HyperText Markup Language)},
    first={Dịch vụ hạ tầng (Infrastructure  as  a  Service - IaaS)}
}

% File định nghĩa các thuật ngữ và từ viết tắt

% Các từ viết tắt
\newacronym{ai}{AI}{Trí tuệ nhân tạo (Artificial Intelligence)}
\newacronym{api}{API}{Giao diện lập trình ứng dụng (Application Programming Interface)}
\newacronym{crud}{CRUD}{Tạo, Đọc, Cập nhật, Xóa (Create, Read, Update, Delete)}
\newacronym{css}{CSS}{Tập tin định dạng tầng (Cascading Style Sheets)}
\newacronym{html}{HTML}{Ngôn ngữ đánh dấu siêu văn bản (HyperText Markup Language)}
\newacronym{http}{HTTP}{Giao thức truyền tải siêu văn bản (HyperText Transfer Protocol)}
\newacronym{https}{HTTPS}{Giao thức truyền tải siêu văn bản bảo mật (HyperText Transfer Protocol Secure)}
\newacronym{jwt}{JWT}{Mã thông báo web JSON (JSON Web Token)}
\newacronym{json}{JSON}{Ký hiệu đối tượng JavaScript (JavaScript Object Notation)}
\newacronym{mvc}{MVC}{Mô hình - Giao diện - Bộ điều khiển (Model-View-Controller)}
\newacronym{orm}{ORM}{Ánh xạ quan hệ đối tượng (Object-Relational Mapping)}
\newacronym{rest}{REST}{Chuyển giao trạng thái đại diện (Representational State Transfer)}
\newacronym{sql}{SQL}{Ngôn ngữ truy vấn có cấu trúc (Structured Query Language)}
\newacronym{ui}{UI}{Giao diện người dùng (User Interface)}
\newacronym{ux}{UX}{Trải nghiệm người dùng (User Experience)}
\newacronym{sms}{SMS}{Dịch vụ tin nhắn ngắn (Short Message Service)}
\newacronym{gpt}{GPT}{Transformer được huấn luyện trước tổng quát (Generative Pre-trained Transformer)}
\newacronym{nlp}{NLP}{Xử lý ngôn ngữ tự nhiên (Natural Language Processing)}
\newacronym{cdn}{CDN}{Mạng phân phối nội dung (Content Delivery Network)}
\newacronym{ssl}{SSL}{Lớp ổ cắm bảo mật (Secure Sockets Layer)}
\newacronym{tls}{TLS}{Bảo mật tầng vận chuyển (Transport Layer Security)}

% Các thuật ngữ
\newglossaryentry{reactjs}{
    name={React.js},
    description={Thư viện JavaScript để xây dựng giao diện người dùng, được phát triển bởi Facebook}
}

\newglossaryentry{nextjs}{
    name={Next.js},
    description={Framework React full-stack hỗ trợ rendering phía máy chủ và tạo ứng dụng web tĩnh}
}

\newglossaryentry{typescript}{
    name={TypeScript},
    description={Ngôn ngữ lập trình được phát triển bởi Microsoft, là phiên bản mở rộng của JavaScript với hệ thống kiểu dữ liệu tĩnh}
}

\newglossaryentry{postgresql}{
    name={PostgreSQL},
    description={Hệ quản trị cơ sở dữ liệu quan hệ-đối tượng mã nguồn mở}
}

\newglossaryentry{prisma}{
    name={Prisma},
    description={ORM hiện đại cho Node.js và TypeScript, cung cấp type-safe database access}
}

\newglossaryentry{tailwindcss}{
    name={Tailwind CSS},
    description={Framework CSS utility-first để xây dựng giao diện người dùng tùy chỉnh nhanh chóng}
}

\newglossaryentry{vercel}{
    name={Vercel},
    description={Nền tảng cloud để triển khai các ứng dụng frontend và serverless functions}
}

\newglossaryentry{openai}{
    name={OpenAI},
    description={Công ty nghiên cứu trí tuệ nhân tạo, phát triển các mô hình ngôn ngữ lớn như GPT}
}

\newglossaryentry{chatbot}{
    name={Chatbot},
    description={Chương trình máy tính được thiết kế để mô phỏng cuộc trò chuyện với người dùng}
}

\newglossaryentry{responsive}{
    name={Responsive Design},
    description={Phương pháp thiết kế web giúp trang web hiển thị tốt trên nhiều thiết bị khác nhau}
}

\newglossaryentry{microservices}{
    name={Microservices},
    description={Kiến trúc phần mềm chia ứng dụng thành các dịch vụ nhỏ, độc lập}
}

\newglossaryentry{serverless}{
    name={Serverless},
    description={Mô hình điện toán đám mây trong đó nhà cung cấp dịch vụ quản lý việc phân bổ tài nguyên máy chủ}
}

\newglossaryentry{encryption}{
    name={Mã hóa},
    description={Quá trình chuyển đổi dữ liệu thành dạng không thể đọc được để bảo vệ thông tin}
}

\newglossaryentry{authentication}{
    name={Xác thực},
    description={Quá trình xác minh danh tính của người dùng hoặc hệ thống}
}

\newglossaryentry{authorization}{
    name={Phân quyền},
    description={Quá trình xác định quyền truy cập của người dùng đã được xác thực}
}

\newglossaryentry{scalability}{
    name={Khả năng mở rộng},
    description={Khả năng của hệ thống để xử lý khối lượng công việc tăng lên}
}

\newglossaryentry{deployment}{
    name={Triển khai},
    description={Quá trình đưa ứng dụng phần mềm từ môi trường phát triển lên môi trường sản xuất}
}

\newglossaryentry{cicd}{
    name={CI/CD},
    description={Tích hợp liên tục/Triển khai liên tục - phương pháp phát triển phần mềm tự động hóa}
}

\newglossaryentry{healthcare}{
    name={Healthcare},
    description={Lĩnh vực chăm sóc sức khỏe, bao gồm các dịch vụ y tế và sức khỏe}
}

\newglossaryentry{telemedicine}{
    name={Telemedicine},
    description={Việc cung cấp dịch vụ y tế từ xa thông qua công nghệ viễn thông}
}

\newglossaryentry{gdpr}{
    name={GDPR},
    description={Quy định bảo vệ dữ liệu chung của Liên minh châu Âu}
}

\newglossaryentry{hipaa}{
    name={HIPAA},
    description={Đạo luật Tính di động và Trách nhiệm giải trình về Bảo hiểm Y tế của Mỹ}
}